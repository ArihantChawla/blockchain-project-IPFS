\chapter{Why IPFS?}

\section{Distributed}
The structure of network decided what kind of applications can be built upon it, what protocols can be used and how much power users have on their own data.

A centralised system allows changing of content quickly but all of the power is present at one location. The end users do not have the same capabilities or control over the data. This also leads to a single point of failure.

A decentralised system divides the duties of the central node to several nodes to which end users connect, thus increasing the resiliency of the network. But this is still less resilient than a completely distributed systems can be.

IPFS for a distributed network, which means that all of the nodes within the system contribute almost equally to the ecosystem. It is a peer-to-peer systems and hence everyone speaks the same protocol. This kind of system is extremely resilient to failures.

In this age where data is a commodity, a distributed system prevents a single entity from controlling a large portion of the data.

\section{Fast}

The current web is location addressed. Our URLs resolve to IP addresses of servers which can be thousands of miles away from our own computers, which is where we fetch our data from. This is the case even when the person sitting next to us has the same data on his machine. IPFS solves this problem by making the web content addressable. This implies that it doesn't matter where the data comes from as long as it is the exact data that we requested. A distributed structure allows data to be present at locations closer to our computers thus reducing communication times and making the web faster.

\section{Offline First}

Think of an example where you are collaborating on a Google Doc with five other friends sitting with you in the same room and suddenly google.com goes down. Now you cannot continue collaborating since all of your computers were talking to google servers. Being a distributed protocol, IPFS allows resiliency to such failures since nodes can directly communicate with one another. It is important to realise here that connectivity is not binary. You can still be connected to other machines in a local area network but be disconnected from the internet. IPFS increases the possibilities even in the scenario of reduced connectivity.