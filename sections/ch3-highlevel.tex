\chapter{High Level Overview / Usage}

Every node in the IPFS ecosystem has a NodeId. It can upload data to the system and retrieve data. IPFS has both a command-line interface as well as a web UI.

Adding directory to IPFS is as simple as:

\begin{minted}{bash}
    ipfs add -r /path/to/dir/
\end{minted}

This recursively adds all the files inside the directory to IPFS and returns an IPFS address for each of them.

To retrieve content, the user needs its address. It can either put the address in the Web UI or use for following terminal command:

\begin{minted}{bash}
    ipfs cat /ipfs/<content_address>
\end{minted}

Let us now look at what's happening under the hood.